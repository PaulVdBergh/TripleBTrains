\hypertarget{group__Z21System}{}\section{System, Status and Version information}
\label{group__Z21System}\index{System, Status and Version information@{System, Status and Version information}}
Class \hyperlink{classTBT_1_1UDPClientInterface}{U\+D\+P\+Client\+Interface} represents the \hyperlink{classTBT_1_1ClientInterface}{Client\+Interface} for U\+DP clients using the Z21 Lan Protocol.

\subsection*{Basics}

\subsubsection*{Communication\+:}

Communication with the Z21 protocol over U\+DP port 21105 or 21106. Control applications on the client (PC, App, ...) should primarily use port 21105. The communication is always asynchronous, i.\+e. between a request and the corresponding response other broadcast messages can occur.

It is expected that each client communicates with the Z21 once per minute, otherwise it will be removed from the list of active participants. If possible, a client should log off when exiting with the L\+A\+N\+\_\+\+L\+O\+G\+O\+FF command.

\subsubsection*{Z21 Datagram layout}

A Z21 datagram, i.\+e. a request or response is structured as follows\+:

\tabulinesep=1mm
\begin{longtabu} spread 0pt [c]{*{3}{|X[-1]}|}
\hline
\rowcolor{\tableheadbgcolor}\PBS\centering \textbf{ Data\+Len }&\PBS\centering \textbf{ Header }&\PBS\centering \textbf{ Data  }\\\cline{1-3}
\endfirsthead
\hline
\endfoot
\hline
\rowcolor{\tableheadbgcolor}\PBS\centering \textbf{ Data\+Len }&\PBS\centering \textbf{ Header }&\PBS\centering \textbf{ Data  }\\\cline{1-3}
\endhead
\PBS\centering 2 bytes &\PBS\centering 2 bytes &\PBS\centering n bytes \\\cline{1-3}
\end{longtabu}

\begin{DoxyItemize}
\item {\bfseries Data\+Len} (little endian)\+:

Total length over the entire data set including Data\+Len, Header and Data, i.\+e. Data\+Len = 2 + 2 + n.
\item {\bfseries Header} (little endian)\+:

Describes the command or protocol group.
\item {\bfseries Data}\+:

Structure and length depend on command. Exact description see respective command.
\end{DoxyItemize}

Unless otherwise stated, the byte order is little-\/endian, i.\+e. first the low byte, then the high byte.

\subsubsection*{}

\subsubsection*{}